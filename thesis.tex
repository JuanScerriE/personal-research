\documentclass[12pt]{article}

\usepackage{amsmath}
\usepackage{amsthm}
\usepackage{amssymb}
\usepackage{mathtools}
\usepackage{latexsym}
\usepackage{cancel}

\usepackage{fancyhdr}

\usepackage{float}
\usepackage{graphicx}
\usepackage{pagecolor}
\usepackage{xcolor}

\usepackage{inputenc}
\usepackage[T1]{fontenc}
\usepackage[default]{lato}
\usepackage{contour}
\usepackage{ulem}


\usepackage[
top=2.50cm,
bottom=2.50cm,
left=2cm,
right=2cm,
marginparsep=0pt,
marginparwidth=0pt]{geometry}

\parindent 0in
\parskip 12pt

\definecolor{Ivory Paper}{HTML}{F7F0DF}

\pagecolor{Ivory Paper}

\newcommand{\floor}[1]{\left\lfloor #1 \right\rfloor}
\newcommand{\ceil}[1]{\left\lceil #1 \right\rceil}
\newcommand{\round}[1]{\left\lfloor #1 \right\rceil}
\newcommand{\abs}[1]{\left\lvert #1 \right\rvert}
\newcommand{\sqref}[1]{[\ref{#1}]}

\DeclareRobustCommand{\ul}[1]{%
	\uline{\phantom{#1}}%
	\llap{\contour{white}{#1}}%
}

\renewcommand{\ULdepth}{1.8pt}
\contourlength{0.8pt}

\theoremstyle{plain}
\newtheorem{thm}{Theorem}
\newtheorem{cor}{Corollary}
\newtheorem{lma}{Lemma}
\newtheorem{prop}{Proposition}
\newtheorem{conj}{Conjecture}

\setlength{\headheight}{15pt}
\pagestyle{fancy}
\renewcommand{\headrulewidth}{0pt}
\lhead{J. Scerri}
\chead{MAT2912 --- Coursework}
\rhead{\thepage}

% TITLE

\title{Graph Theory\\
\vspace{0.75em}\textbf{Personal Research}}

\date{\today}

\author {{\textbf{Juan Scerri}}\\
B.Sc. (Hons)(Melit.) Computing Science and Mathematics (Second Year)}

\begin{document}

\maketitle % Print the title page

\thispagestyle{empty} % Suppress headers and footers on the title page

\begin{conj}[Erd\H os–Gy\` arf\` as Conjecture]

Every simple graph $G$ with minimum degree $3$ contains a simple
cycle whose length is a power of two.

\end{conj}

At first glance there does not seem any clear way to approach
the problem. So instead we will define a weaker version of the
conjecture. Hopefully, the machinery developed for solving the
weaker conjecture will prove to be useful when tackling the full
conjecture.

\begin{conj}

Every simple graph $G$ with minimum degree $3$ contains a simple
cycle whose length is even.

\end{conj}

We will try to approach this by assuming that the all cycles are
of odd length. Now we need to look at the most general way
cycles of odd length can interact with each other.




\begin{thm}
\end{thm}

\begin{conj}
\end{conj}

\begin{lma}
\end{lma}

\begin{prop}
\end{prop}

\begin{corollary}
\end{corollary}

\end{document}

